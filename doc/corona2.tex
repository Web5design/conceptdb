\documentclass{article}
\usepackage{palatino}
\usepackage{fullpage}

\title{CORONA redesign --- March 2011}
\author{Rob Speer}

\begin{document}
\maketitle

CORONA, as I described it in my January AAAI submission, did not work as
planned. Although I didn't discover this in time to finish the paper, its
emergent effects on large networks turned out to undesirable.  For example:
either (without normalization) agents that cast a lot of votes accumulate lots
of power, or (with normalization) every node can cause huge swings of
reliability in its sparsely-connected neighbors.

On real ConceptNet data, the first case leads to the unreliable Verbosity
having thousands of times more influence than other knowledge sources, and the
second case means that good critics manage to negate themselves out of
existence.

The main sources of these problems were:
\begin{itemize}
\item PageRank-style influence congregates in highly-connected areas of the
  graph. When influence is the same thing as reliability, there is no
  meaningful global scale for reliability.
\item Negations require subverting the mathematical machinery that made the
  system work in the first place. The combined effect of many negations can
  send the model into Bizarro Minus World, yielding results that make the
  opposite of sense.
\item Negations also created ``retribution'': if node A downvotes node B, that is the
  same as node B downvoting node A. If a node is a person, that person would
  never want to downvote anything.
\end{itemize}
\section{Weighted averaging}

Terminology: we have nodes with different amounts of reliability casting votes
about different amounts of reliability. Let us refer to these as the \emph{node
weight} and the \emph{edge weight} respectively. Then, the reliability of a
node comes from two sources:

\begin{itemize}
\item The average edge weight of its predecessors, weighted by their node weight
  (How reliable do other nodes think you are?)
\item The average node weight of its successors, weighted by their edge weight
  (When you say whether other nodes are reliable, how good is your word?)
\end{itemize}

Notice the exchange of ``node'' and ``edge'' there: these effects are dual to
each other. 

Suppose node A upvotes node B (says that B should have high reliability). A
dual message comes back from node B about how good this vote was, based on the
overall score node B gets, and this will be averaged into the reliability of
node A. This is the behavior we want in a very common case: agreement leads to
reliability.

Now suppose that node A downvotes node C, saying that C should have reliability
0. C does not drag down A with it, because the dual message from node C has
\emph{weight} 0.

\subsection{Weights vs. probabilities}

Although these numbers fall between 0 and 1, I still do not intend to treat
them as probabilities, because the events described have no known probabilistic
model, and because reconciling different assessments of probability for the
same event is an unnecessarily complex problem. These numbers should instead be
considered as fuzzy truth values.

These values can be considered directly as fuzzy logic truth values, instead of
requiring mapping through a function $f$.

The justification graph, as before, may contain disjunctions or conjunctions of
justifications. Here is the mapping between those and fuzzy logic:

\begin{itemize}
\item Disjunctions of justifications are combined in a weighted average, as
described above.
\item Conjunctions are implemented using the Hamacher product $H(a, b) =
ab/(a+b-ab)$, as in the original paper. The advantage of the Hamacher product
over the standard arithmetic product is that conjunctions of low-reliability
nodes are punished only linearly, not quadratically. $H(a, a) = o(a)$ as $a
\rightarrow 0$, while $a \cdot a = o(a^2)$.
\item Negations are no longer necessary as a separate mechanism; a node can decrease the reliability of another node by giving it a vote of 0 or a sufficiently low number.
\end{itemize}

\section{Combining justifications and a prior}

Nodes start with a defined level of reliability, which can be set to an
arbitrary value. Each node starts with the equivalent of one vote for it at
that level of reliability. Because the weighted average and the Hamacher
product are approximately linear, changing the prior should only change the
scale factor of the results.
\end{document}
